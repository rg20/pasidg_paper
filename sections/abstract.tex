%% abstract
\begin{abstract}
%We describe the application of the high performance, high order discontinuous Galerkin methods for faster than real-time simulations of tsunami wave propagation. We adapt the methods described in \cite{gandham2014swe} and validate them with the tidal gauge recordings of 2004 Indian Ocean tsunami event. Our implementations use OCCA \cite{medina2014occa, occa2}, a portable and extensive multi-threading library.  OCCA\ library alleviates the need to completely re-design the solver to keep up with constantly evolving parallel programming models and hardware architectures. We present performance results for the real world tsunami simulations leveraging multiple different multi-threading APIs on GPU and CPU targets. 
Modeling the tsunami wave propagation is of great importance in developing warning systems and disastrous management planning. High fidelity models require very large scale computations. The requirement of running such simulations many times in stochastic analysis demands efficient and scalable simulators. In this paper, we consider two dimensional shallow water equations for modeling the tsunami wave phenomenon, a high order discontinuous Galerkin method   for the numerical simulation, and develop efficient algorithms that are tailored for SIMD architecture seen in modern GPUs. By adopting physically motivated multi-rate time stepping method and carefully designed implementations, we can simulate a realistic 10 hrs of 2004 Indian Ocean tsunami  in one minute on a single GPU workstation. We further demonstrate the efficiency of the developed simulator with 2011 Japan tsunami in global ocean. We use OCCA, an extensible multi-threading API that allows us to run the simulation on the many computing hardware architectures supporting any of CUDA, OpenCL, or OpenMP programming models.  


\end{abstract}
