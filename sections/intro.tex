\section{Introduction}
\label{sec:intro}

$2004$ Indian Ocean tsunami, one of the deadliest tsunami in history, killed about $200,000$ people and cause a damage worth of billions of dollars, in $14$ countries in the Indian\ Ocean shorelines. Since then, there has been growing research and development in building tsunami warning systems around the world. These systems are designed to forecast tsunami arrivals and their impact in the coastal regions, hours before the waves arrive. 

 These forecasts involve simulating the wave propagation of tsunami waves over  large distances accurately in a timely manner, after a single or multiple earthquakes are detected in the ocean. This is challenging because of 
the complexity of the wave propagation due to the presence of largely varying length scales, varying bathymetry, and nonlinear effects near the shore. Stable, accurate and efficient algorithms are of great interest for these applications.

In this work, the two-dimensional shallow water equations are used to model the tsunami wave propagation. These equations are commonly accepted for modeling tsunami propagation \cite{george2006finite}. A high order discontinuous Galerkin (DG) discretization method on unstructure triangular meshes, is adapted for obtaining a numerical solution of the PDE. DG  methods are well suited for shallow water equations and wave propagation, in general, due to their good phase propagation properties, conservation properties, and capability of handling complex shoreline geometries. These methods achieve high order accuracy on irregular meshes by a high order polynomial representation of fluid properties in each element of the mesh. Nevertheless, these methods are computationally expensive compared to methods like finite volume, finite difference, and linear finite elements. However,  the recent advances in computing hardware architectures such as GPUs and many-core CPUs make it possible to achieve simulations in a reasonable time if the algorithms are well tuned to take advantage of the architecture trends. 

In recent years, several programming models such as OpenMP, CUDA, and OpenCL have been evolved alongside the hardware architectural trends. Because of this, the algorithms have to implemented in more than one programming (or multi-threaded) model to take advantage of various hardware architecture. To alleviate the burden of implementing in several programming models, our implementations use extensive and unified many-core programming library called OCCA \cite{occa2}. This gives us flexibility in choosing the most efficient multi-threading model for a given hardware architecture, without re-writing the code.
 
 This paper is organized as follows: In Section \ref{sec:governing}, the governing PDE model is introduced. In Section \ref{sec:discretization}, the discontinuous Galerkin space discretization and the multi-rate time stepping scheme are discussed briefly. In Section \ref{sec:datasets}, the data sets used for the simulation of Indian Ocean tsunami, the construction of initial conditions, boundary conditions, and implementation details of friction terms are elaborated. In Section \ref{sec:results}, the numerical validation results are presented along with the performance results. 
